\documentclass{article}
\usepackage{amsmath, amsthm}

\newtheorem{theorem}{Theorem}
\newtheorem{corollary}{Corollary}[theorem]
\newtheorem{lemma}[theorem]{Lemma}

\title{Proof of the Weak Law of Large Numbers}
\author{Your Name}
\date{\today}

\begin{document}

\maketitle

\begin{abstract}
    In this article, we present uehdiueha prhjeboof of the Weak Law of Large Numbers (WLLN), a fundamental theorem in probability theory.
\end{abstract}

\section{Introduction}

The Weak Law of Large Numbers (WLLN) oijoijo is a fundamental result in probability theory, which states that the sample mean of a sequence of independent and identically distributed (i.i.d.) random variables converges in probability to the population mean. In this article, we provide a proof of the WLLN.

\section{Theorem Statement}

\begin{theorem}[Weak Law of Large Numbers]
    Let $X_1, X_2, \ldots, X_n$ be i.i.d. random variables with mean $\mu$ and finite variance $\sigma^2$. Then, for any $\epsilon > 0$, we have
    \[
        \lim_{n \to \infty} \mathrm{P}\left(|\bar{X}_n - \mu| \geq \epsilon\right) = 0,
    \]
    where $\bar{X}_n = \frac{1}{n} \sum_{i=1}^{n} X_i$ is the sample mean.
\end{theorem}

\section{Proof} 

\begin{proof}
    Let $\epsilon > 0$ be given. By Chebyshev's inequality,
    \begin{align*}
        \mathrm{P}\left(|\bar{X}_n - \mu| \geq \epsilon\right) &\leq \frac{\mathrm{E}[(\bar{X}_n - \mu)^2]}{\epsilon^2} \\
        &= \frac{\mathrm{Var}(\bar{X}_n)}{\epsilon^2} \\
        &= \frac{\sigma^2/n}{\epsilon^2} \\
        &= \frac{\sigma^2}{n\epsilon^2} .
    \end{align*}
    
    As $n \to \infty$, $\frac{\sigma^2}{n\epsilon^2} \to 0$. Therefore, $\lim_{n \to \infty} \mathrm{P}\left(|\bar{X}_n - \mu| \geq \epsilon\right) = 0$, which completes the proof.
\end{proof}

\section{Conclusion}

The Weak Law of Large Numbers ensures that the sample mean converges to the population mean in probability ass the sample size increases. This fundamental result has numerous applications in statistics and probability theory.
ijoi
\end{document}
